\section{Conclusion}
\label{sec:conclusion}

I have presented
the power scaling behavior
of our wafer-fused
$1300\,\mathrm{nm}$ waveband
VECSEL device
in the thin disk (flip-chip)
heat dissipation scheme.
This scaling was found
to be below
the expected behavior
for disk lasers,
but are in line with
other published findings.
It is assumed
to be a result
of the sub-area scaling
of the thermal resistance --
whose scaling behavior
was also demonstrated
in this report.

Changing the pump configuration
to incorporate an achromatic lens,
resulted in significantly
improved output performance.
In this configuration
I have investigated
upon two optimization strategies:
(i) anti-reflectance (AR) coating,
(ii) improved reflectivity
of the Au bonding layer
between DBR and heat sink.
This second strategy
resulted in record output power
of $8.5\,\mathrm{W}$
for a heat sink temperature
of \degr{5}.
The AR approach we had to
abandon due to insufficient
coating quality.
The found difference
in output performance
as a result of
changed pump conditions
suggests the pump beam profile
to be more important
than usually assumed.
As such,
a carefully prepared pump
is likely to improve
a VECSEL's performance,
while keeping the chip design.

The longest emitted wavelength,
resulting from the heating
of the VECSEL device
due to increased pump power,
appears to tend to a value
that is independent of
the applied heat sink temperature.
This finding suggests
the power roll over
of our device
to occur at
an intrinsic critical temperature;
this property
was thus far not confirmed
for our structure.
It allows
to determine
the thermal resistance
in a way
that doesn't rely
on a spectral shift
of the emitted light.

Furthermore,
I have presented
the reflectivity
off our VECSEL structure
to change
for higher pump power.
This behavior
has so far not been reported.
By taking into account
these measured reflectivity values --
instead of assuming
the reflectivity
to remain at its low pump reflectivity level --
also the found conversion efficiency
of our device
showed record results;
up to nearly $60\,\%$ for
\degr{5} heat sink temperature.

Beside these direct results,
I have assessed
two methods
to experimentally
determine the thermal resistance
of a VECSEL,
that can be measured
simultaneously with
regular light-light characteristics.
I have to conclude
these methods
not to yield
a good figure of merit.
The identified dependencies
inevitably link
the resulting thermal resistance
with the setup it was measured with.
I have showed these shortcomings
by means of
experiment --
the spectral behavior
varied as a result
of the pump conditions --
and numerical considerations.

The presented
maximum output power
reached during this project is
limited by pump power,
heat sink temperature stability,
and the lack of control
over the pump conditions --
most notably
the pump distribution.
In addition,
the investigated samples
show a high reflectivity,
such that a considerable amount
of pump is wasted.
For future works
I suggest to invest
time and resources
directed at these domains.
The sample reflectivity
could also be exploited
by using a mirror arrangement
that recycles
the reflected power
and redirects it
onto the pumped spot.

This report acts
as basis
to reach out
to other groups,
who have already
heavily invested
in numerical approaches:
if the improvements in output performance
indeed can be partially attributed
to the change in pump profile,
more effort
should be put into investigating
this aspect
for VECSEL applications.
One of the key arguments
of VECSELs
is their capability
to convert low-cost pump light
into high quality laser emission.
If this conversion
can be improved
by simple modifications
in the pump channel,
such findings
may benefit
a large variety of VECSEL applications.
Additionally,
it will be interesting
to see the detailed reflectivity behavior
of other devices.

As a last point,
the presented measurements
have demonstrated
the high degree
of sample integrity:
our wafer-fused devices
did not show
any signs
of degradation.
And this despite the fact
that the samples were heated up
and cooled down
over a broad range of temperatures;
the samples were exposed to
beyond roll over pump powers;
the pump irradiation
did not occur
in a smooth ramp order
but were randomly sampled
instead.
These findings
demonstrate a high device quality,
obtainable by wafer fusion.
