\subsection{Power-scalability}
\label{sec:rth:scaling}

The gain section of a VECSEL
is very thin compared to
the dimensions of the optical pump spot.
This gain region
is heated
due to the difference in photon energy of pump and output
and due to non-radiative recombinations.
In other words,
heat is extracted
over a short path
with respect to the pump spot size,
and the resulting
heat flow is
approximatively one dimensional.
Lateral cooling is
therefore
not as relevant
for the operation of such a device.
One dimensional heat extraction means
the device is power scalable:
increasing the pumped area
does not introduce
significantly additional heating.
The output power
can be enhanced
by enlarging the pumped area,
so more of the gain material
is stimulated
into emission.
This is the expected behavior
of disk lasers.
\cite{Tropper2006,Lindberg2005,Le1991}.

Several effects
hinder the real device
to live up to these expectations.
Bedford et al. \cite{Bedford2005} report
a limit to this scalability:
there is a critical diameter
above which amplified spontaneous emission (ASE)
and lateral lasing
become relevant.
These effects introduce additional losses
which inevitably limit the output scaling.
The pump beam shape
was found to
limit the power scaling further \cite{Chernikov2010}.
Furthermore,
larger beam spots
risk to be more susceptible
to surface impurities
and non-radiative defects \cite{Korpi2010}.

If the thermal resistance $\Rth$
were to power scale,
its value would have to decrease
at an inverse rate of the pumped area, $w^{-2}$.
As discussed by Giet et al. \cite{Giet2008},
this is not the case.
Instead,
the decrease in $\Rth$ follows more closely
the radius (diameter) of the pump spot
than the area, $w^{-1}$.
This behavior is apparent when we plot
spot size versus thermal resistance
in a log-log plot.
The slope in such a plot
is closer to $-1$ than to $-2$.

For the sake of completeness,
the thermal resistance can be fitted with
the empirical formula \cite{Giet2008}
\begin{equation}
\Rth(w) = a_1+ \frac{a_2}{w} \left( 1-\frac{w}{a_3} \right)^{1.5}.
\label{eq:Rth_empirical}
\end{equation}
This formula was adopted
form a fit originally developed for VCSELs \cite{Nakwaski1992}.
In the original form
the ratio within the brackets
represented the ratio between an inner and outer diameter
of the VCSEL.
Working with VECSELs we lack such a definition.
As such,
it is not clear
what parameters $a_{\{1,2,3\}}$ are supposed
to tell us about the VECSEL under test.
So far,
this description is not yet widely used
by the scientific community.
