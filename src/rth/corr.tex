\subsection{Remarks on experimental access and corrections}

Hader et al. \cite{Hader2013}
have pointed out,
the used definition of
dissipated power (\ref{eq:dissip})
ignores a relevant loss channel.
Namely,
beside the reflected,
emitted,
and dissipated part
there are additional non-heating losses
the incident pump converts into.
These additional losses come from
non-heating spontaneous emission
and surface-scattering.
Without these
we overestimate $\Rth$.
The losses due to surface-scattering
are particularly pronounced
when using an output coupler
with low out-coupling losses.

From a theoretical point of view
the comments made by
Hader et al. \cite{Hader2013}
may be relevant.
However, for the experimental approach
they're impractical
as the invoked calibration
is tedious.
Especially,
if the main objective in monitoring $\Rth$
is to obtain a figure of merit
for the thermal management
of the VECSEL structure.
As a remedy,
Hader et al. suggest
using an output coupler
with higher out-coupling losses.
The scattering losses
would be less pronounced.
In this report,
I ignore these additional effects
and work with (\ref{eq:dissip}).
Consequentially,
the stated values for the determined $\Rth$'s
are likely to overestimate
the true value.