\section{Introduction}
\label{sec:intro}

A vertical-external-cavity
surface-emitting laser
(VECSEL)
is a special type
of semiconductor laser.
It converts
low cost diode pump light
into high beam quality
laser emission.
VECSELs combine
the benefits known from
solid state lasers,
such as high output power
with high beam quality,
with the flexibility
of semiconductor lasers
and their coverage
of bandgap engineered
emission wavelength
\cite{Tropper2006,Ranta2014OptLett,Kemp2008,Chernikov2010}.

The advantages of this technology
made it attractive for a
wide variety of applications,
such as
optical communication,
spectroscopy,
medical applications
(both on detection and surgical side).
VECSELs achieve
high intra-cavity power
which is ideal
for non-linear optics.
They can thus also be used
to reach visible light
via higher harmonics,
which itself opens the field for
laser TV and projectors.
And frequency mixing
also enables VECSELs
to operate within the domains of
microwave photonics,
LIDAR, and THz sources 
\cite{Ranta2014OptLett,Calvez2009,Sirbu2014OptExp,Baili2009,Lukowski2015,Bedford2005}.

The $1300\,\mathrm{nm}$ waveband
to date lacks options
for high power laser sources.
This range is particularly interesting
\cite{Ranta2014OptLett,Sirbu2014OptExp}
to pump and laser process
materials
beyond the reach
of conventional
sub-micron light sources.
Furthermore,
the frequency doubled
$650\,\mathrm{nm}$
red light
fills an important gap
for laser projection
and spectroscopy.
Additionally,
this wavelength range
benefits from
pre-existing infrastructure
and material development
as a result of
heavy investments of ICT.

Our VECSEL device
is composed of
a gain region
grown by metallorganic vapor phase epitaxy
(MOVPE)
on a InP substrate
incorporating
AlGaInAs quantum wells.
The distributed Bragg reflector (DBR)
was grown by molecular beam epitaxy (MBE).
These two elements were wafer-fused
and Au-Au-bonded
onto chemical vapor deposited (CVD) diamond,
which itself was attached
onto a water-cooled heat sink.
I present
more on VECSEL basics
and about the wafer fusion
in the following section.
The gold bonding interface
introduces beneficial
thermal and optical properties;
this will play a role
as we try to improve
the performance of our device.
The manufacturing details
are beyond the scope of this project,
they can be found in
\cite{Ranta2014OptLett,Sirbu2014SPIE}.

Numerical tools
can be used
to predict some of
the VECSEL behaviors.
Their usage,
however,
is fairly limited.
This scarcity originates
from the fact that
in a VECSEL there are
many effects to be considered:
attempts have been made
to calculate the
absorption and gain of the quantum wells
using semiconductor Block equations,
considering spontaneous emission
using semiconductor luminescence equations,
and taking into account
carrier losses
using quantum Boltzmann scattering equations
\cite{Hader2011}.
All of these steps
rely on exact material parameters
for a broad range of
doping concentrations,
operation temperatures,
and wavelengths.
The dimensionalities
of a VECSEL chip
along the plane
and growth axes
are different by orders of magnitude.
This complicates
the meshing procedure
for finite element analyses,
which is a problem
the thermal simulations
in particular
suffer from. 

On top of that,
the optical pump conditions
are usually ignored
in numerical models.
The pump spot profile
is considered
to be one of two
simplest options to implement:
a Gaussian
or a flat-top profile;
under normal incidence
\cite{Ranta2014OptLett,Kemp2008,Hader2011,Kemp2005}.
Meanwhile,
a more realistic pump profile
would resemble something in between
these two extreme forms,
since the pump
is usually delivered
via a multi-mode fiber,
under non-normal incidence
\cite{Tropper2006,Chernikov2010,Heinen2012el}.

As such,
the simulations provide
valuable insights
in general trends
of VECSEL behavior
depending on particular parameters
\cite{Hader2011,Vetter2012}.
But one should be very careful
using the numerical results
as a fit
onto experimental results.

The goal of this project was
to investigate --
experimentally --
the performance
of our VECSEL structure
under high power operation.
This builds
on previously published work
\cite{Ranta2014OptLett,Sirbu2014OptExp}.
Our idea is to learn about
the power-scalability
of our device,
as well as to assess
certain metrology processes.
Based on the gained insights
we then implemented
two proposed strategies
\cite{Hader2011}
to further improve the performance.
The benchmark was set
by the current record output power
for the $1300\,\mathrm{nm}$ waveband
of $7.1\,\mathrm{W}$
for a $300\,\mu\mathrm{m}$ pump spot diameter
\cite{Sirbu2014OptExp}.

This report starts
by discussing
the concept of VECSELs
from a general point of view.
In section~\ref{sec:basics},
I look at
the basic elements involved
in order to obtain
an operating VECSEL device.
Since VECSELs
are rich in details,
I have to
restrain the introduction to 
the aspect of thermal management.
The way heat flows
throughout the structure
is vital for its operation.
In section~\ref{sec:rth},
I introduce
the concept of thermal resistance;
a parameter that helps us
to understand
the heat extraction.
This includes
our understanding of this quantity
from an experimental
and numerical --
in section~\ref{sec:comsol} --
point of view.

Given this general understanding
of the important quantities,
in section~\ref{sec:exp}
I present our
specific experimental setup
and measurement routine
aimed to extract said parameters.
This leads to a discussion
and interpretation
of the obtained measurement results
in section~\ref{sec:eval}.
I close this report
with a summary
and conclusion;
and an outlook on
how to advance the project
given the current state
of the VECSEL characterization.
The appendices
provide additional insights
for some of the discussed points --
but that do not belong to
the core report.
