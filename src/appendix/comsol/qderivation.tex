\subsection{Heat source $Q$}
\label{app:comsol_deriv:Q}

In order to understand
how to investigate
the temperature distribution
within a VECSEL structure,
we have to understand
what input to provide to COMSOL
to solve the heat equation (\ref{eq:app:heateq}).
Our optical pumping represents
a heat source $Q$.
In this subsection
I present what this quantity looks like,
and what influence the different parameters --
most notably, the beam profile --
have.

The pump beam
is assumed to be
incident antiparallel to the $z$-axis
(here referred to as from the top).
We assume further
it is of Gaussian profile --
section~\ref{sec:comsol:beamprofile}
explains how to incorporate
different pump profiles.
In this configuration the heat source
associated with
each layer $j$ of the structure is given as
\cite{Kemp2008}
\begin{equation}
Q_j = \frac{2P}{\pi w^2} \eta_j\alpha_j \e^\frac{-2r^2}{w^2} \e^{-\alpha_j(z_{0j}-z)} \e^{-\sum_{i<j}\alpha_i t_i}.
\label{eq:app:heatsrc}
\end{equation}
The layers are counted from the top down --
the sum $\sum_{i<j}$ includes the layers on top
and ignores those below the layer of interest
(in that case $j$).
The meaning of the single parameters are listed in Tab.~\ref{tab:heatsrc}.
Furthermore we recognize
\begin{equation}
P'=P\e^{-\sum_{i<j}\alpha_i t_i}
\end{equation}
to represent the power still remaining after layer $1,2,\ldots,j$.
And
\begin{equation}
A=P'\eta_j
\end{equation}
corresponds
to the absorbed power in layer $j$;
which heats by factor $\alpha_j$.

\begin{table}[h]
\centering
\caption{Meaning of parameters used in (\ref{eq:app:heatsrc}).}
\begin{tabular}{llc}
\hline
Parameter & Explanation & Unit \\
\hline
$P$ & pump power & $\mathrm{W}$ \\
$w$ & pump beam $1/e^2$ radius & $\mathrm{m}$ \\
$\alpha_j$ & absorption coefficient of layer $j$ & $\mathrm{m}^{-1}$ \\
$r$ & radial coordinate & $\mathrm{m}$ \\
$z$ & axial coordinate & $\mathrm{m}$ \\
$z_{0j}$ & coordinate of top of layer $j$ & $\mathrm{m}$ \\
$t_j$ & thickness of layer $j$ & $\mathrm{m}$ \\
$\eta_j$ & heat loading fraction in layer $j$ & - \\
\hline
\end{tabular}
\label{tab:heatsrc}
\end{table}

The Gaussian beam assumed in (\ref{eq:app:heatsrc})
has an E-field \cite{QE}
\begin{equation}
E(r,z) \propto \frac{w_0}{w(z)} \exp(-\frac{r^2}{w^2(z)}) \exp(-i\Phi).
\label{eq:gaussE}
\end{equation}
The intensity of such a beam is proportional to the square modulus
\begin{align}
I(r,z) &\propto |E(r,z)|^2, \\
I(r,z) &= I_0 \left(\frac{w_0}{w(z)}\right)^2 \exp(-\frac{2r^2}{w^2(z)}).
\label{eq:gaussI}
\end{align}
This is where the factor $2$
in the exponent of (\ref{eq:app:heatsrc})
comes from.
The over all power
contained within the cross section
is constant, $P$.
Hence follows
the last part of (\ref{eq:app:heatsrc})
\begin{align}
P &= \int\limits_0^{2\pi}\d\varphi\int\limits_0^\infty r\d r I(r,z) \\
 &= 2\pi I_0 \left(\frac{w_0}{w(z)}\right)^2 \int\limits_0^\infty r \d r \exp(-\frac{2r^2}{w^2(z)}) \\
 &= 2\pi I_0 \left( \frac{w_0}{w(z)}\right)^2 \left[ -\frac{w^2(z)}{4} \exp(-\frac{2r^2}{w^2(z)}) \right]_0^\infty \\
 &= \frac{\pi}{2} w_0^2 I_0, \\
\Rightarrow\quad I_0 &= \frac{2P}{\pi w_0^2}.
\end{align}
