\subsection{Implementation of our VECSEL structure}
\label{app:comsol_deriv:impl}

In this subsection I discuss the parameters
relevant for the actual COMSOL implementation,
taking our structure as an example
\cite{Ranta2014OptLett}.
A schematic of our VECSEL structure
is depicted in Fig.~\ref{img:design}.
Its fabrication details can be found in
\cite{Ranta2014OptLett,Sirbu2014SPIE}.
A summary of the used parameters --
in order to reproduce the presented results --
is given in appendix~\ref{app:comsolparams},
Tab.~\ref{tab:comsolparams}.

For the thermal conductivity
we have to consider
its spatial anisotropy.
The radial and axial thermal conductivity
we find with
\cite{Vetter2012,Lindberg2005,Osinski1993}
\begin{align}
k_r &= \frac{\sum_i k_i t_i}{\sum_i t_i} \\
k_z &= \frac{\sum_i t_i}{\sum_i \frac{t_i}{k_i}},
\label{eq:radaxthermcond}
\end{align}
with the actual values calculated from \cite{ioffe}
\begin{equation}
k_{\mathrm{Al}_{x}\mathrm{Ga}_{1-x}\mathrm{As}} =
0.55-2.12x+2.48x^2\,\mathrm{Wcm}^{-1}\mathrm{K}^{-1}.
\label{eq:k_AlGaAs}
\end{equation}
Before we employ
the extracted values of
a bulk Al$_{x}$Ga$_{1-x}$As material,
we divide them in half.
This way we account for
the empirically observed decrease
in thermal conductivity
in a layered structure such as a DBR
\cite{Piprek1998,Ranta2014APL}.

The absorption coefficient $\alpha_\mathrm{g}$ 
is chosen such that
it matches with the measured single pass absorption of $50\,\%$
\cite{Ranta2014OptLett}:
\begin{equation*}
\e^{-\alpha_\mathrm{g}t_\mathrm{g}}=0.5.
\end{equation*}

As a last important note we have to account
for the reflected beam:
it is absorbed in the gain section once again.
Hence we implement a second heat source in this gain layer
according to (\ref{eq:heatsrc})
\begin{equation}
Q_{\mathrm{g}_\mathrm{refl}} = \frac{2P}{\pi w^2} \eta_{\mathrm{g}_\mathrm{refl}}\alpha_g \e^\frac{-2r^2}{w^2} \e^{-\alpha_g(z_{0g}-t_g+z)}
 \e^{-\alpha_\mathrm{InP} t_c -\alpha_\mathrm{g} t_\mathrm{g} -2\alpha_\mathrm{InP}t_\mathrm{sp} -2\alpha_\mathrm{d}t_\mathrm{d}}.
\label{eq:heatsrcrefl}
\end{equation}
In this we have the expression
\begin{equation}
\eta_{\mathrm{g}_\mathrm{refl}} = \eta_\mathrm{g} (1-\eta_\mathrm{au}),
\label{eq:etagrefl}
\end{equation}
that assumes the fraction of reflected light
to be the residual light of what was not absorbed by the gold layer.
This remaining part heats the layer with efficiency $\eta_\mathrm{g}$.
Furthermore we see the sign in the exponent
\begin{equation*}
\e^{-\alpha_g(z_{0g}-t_g+z)}
\end{equation*}
to be positive.
This comes from the fact that in reflection
the beam travels in the opposite direction.
